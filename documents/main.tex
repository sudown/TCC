\documentclass[12pt]{article}
\usepackage{ArtigoIFPE}
\addbibresource{referencias.bib}

% --- INFORMAÇÕES DO TRABALHO ---
\title{DETECÇÃO DE WEBSITES DE PHISHING UTILIZANDO MACHINE LEARNING: uma análise comparativa de algoritmos de classificação}
\titleEng{PHISHING WEBSITES DETECTION USING MACHINE LEARNING: a comparative analysis of classification algorithms}

\autora{José Mirosmar}
\emaila{jmss6@discente.ifpe.edu.br}
\orientador{João Almeida e Silva}
\emailOrientador{joao.almeida@belojardim.ifpe.edu.br}

\campus{Belo Jardim}
\curso{de Bacharelado em Engenharia de Software} 
\data{14 de setembro de 2025}

\begin{document}

\maketitle

\thispagestyle{plain}

% --- RESUMO E ABSTRACT (reescritos no tempo passado) ---
\section*{Resumo}
\noindent O crescente número de ameaças cibernéticas, com destaque para os ataques de phishing, representa um risco significativo para a segurança de dados de usuários e empresas. Este trabalho desenvolveu e avaliou modelos de Machine Learning como uma abordagem para a detecção automática e inteligente de websites de phishing. O objetivo foi comparar a eficácia dos algoritmos de classificação Random Forest e Support Vector Machine (SVM) na identificação de URLs maliciosas. A metodologia foi baseada em uma abordagem quantitativa, utilizando um dataset público da plataforma Kaggle. Os dados foram pré-processados e analisados, servindo de base para o treinamento e teste dos modelos. Os resultados demonstraram que o modelo Random Forest alcançou uma acurácia de 98.20\%, superando significativamente o desempenho do SVM, que obteve 86.35\%. Concluiu-se que, para o conjunto de dados e as condições avaliadas, o Random Forest é uma abordagem mais robusta e eficaz para o problema proposto, contribuindo para o avanço das pesquisas em segurança da informação.

\palavraschave{Segurança da Informação. Phishing. Machine Learning. Random Forest.}

\section*{Abstract}
\noindent The increasing number of cybersecurity threats, especially phishing attacks, poses a significant risk to the data security of users and companies. This work developed and evaluated Machine Learning models as an approach for the automatic and intelligent detection of phishing websites. The objective was to compare the effectiveness of the Random Forest and Support Vector Machine (SVM) classification algorithms in identifying malicious URLs. The methodology was based on a quantitative approach, using a public dataset from the Kaggle platform. The data was pre-processed and analyzed to serve as a basis for training and testing the models. The results demonstrated that the Random Forest model achieved an accuracy of 98.20\%, significantly outperforming the SVM model, which obtained 86.35\%. It was concluded that, for the dataset and conditions evaluated, Random Forest is a more robust and effective approach for the proposed problem, contributing to the advancement of research in information security.

\keywords{Information Security. Phishing. Machine Learning. Random Forest.}

\vspace*{20pt} \hrule height 1.5pt

% --- CORPO DO TCC FINAL ---

\section{Introdução}

% [TODO: Revisar este texto, mas ele já está 90% pronto do pré-projeto]
A onipresença da internet transformou a sociedade, mas também introduziu novas vulnerabilidades. Dentre as ameaças cibernéticas, o \textit{phishing} se destaca como um dos ataques mais prevalentes e danosos, visando enganar usuários para que revelem informações sensíveis, como credenciais de acesso e dados financeiros. A crescente sofisticação destes ataques torna a detecção manual insuficiente, criando uma demanda por soluções automáticas e inteligentes.

Neste contexto, o Machine Learning (ML) surge como uma abordagem promissora \parencite{hastie2009elements}. Ao treinar algoritmos para reconhecerem os padrões característicos de URLs maliciosas, é possível desenvolver ferramentas capazes de identificar ameaças em tempo real com alta precisão, como demonstrado em estudos recentes na área \parencite{mandadi2022}.

\subsection{Justificativa}
% [TODO: Esta seção geralmente se mantém igual ao pré-projeto.]
A relevância deste trabalho reside no potencial de mitigar os impactos negativos do phishing. A criação de modelos de detecção eficazes contribui diretamente para a segurança de usuários, a proteção da reputação de empresas e a redução de perdas financeiras. Academicamente, a pesquisa avança o estado da arte ao comparar algoritmos específicos para este problema, gerando conhecimento aplicável tanto no meio industrial quanto no científico.

\subsection{Problema de Pesquisa}
% [TODO: Esta seção geralmente se mantém igual ao pré-projeto.]
A questão central que norteia este trabalho é: De que forma e com qual eficácia os algoritmos de Machine Learning, especificamente Random Forest e Support Vector Machine, podem ser aplicados para a detecção automática de websites de phishing com base em características de suas URLs?

\subsection{Objetivos}
% [TODO: Esta seção geralmente se mantém igual ao pré-projeto.]
\subsubsection{Objetivo Geral}
Desenvolver e avaliar a performance de modelos de Machine Learning para a detecção de websites de phishing.

\subsubsection{Objetivos Específicos}
\begin{itemize}
    \item Realizar uma revisão bibliográfica sobre os temas de phishing e algoritmos de classificação;
    \item Analisar e pré-processar um conjunto de dados públicos sobre o tema;
    \item Implementar e treinar os modelos de classificação Random Forest e Support Vector Machine;
    \item Comparar o desempenho dos modelos utilizando métricas de avaliação como acurácia, precisão e recall.
\end{itemize}


\section{Fundamentação Teórica}
\label{sec:fundamentacao}
% [TODO: Escrever este capítulo com base na sua pesquisa. Use subseções.]
\subsection{Phishing: Conceitos e Técnicas}
% [TODO: Explicar o que é phishing, como funciona, tipos de phishing, etc.]

\subsection{Machine Learning e Aprendizado Supervisionado}
% [TODO: Explicar os conceitos básicos de ML, focando em classificação.]

\subsection{Algoritmos de Classificação}
% [TODO: Explicar de forma conceitual como o Random Forest e o SVM funcionam.]

\subsection{Trabalhos Correlatos}
% [TODO: Resumir 2 ou 3 artigos que fizeram algo parecido com o seu trabalho, como o de Mandadi et al. (2022).]


\section{Metodologia}
\label{sec:metodologia}
% [TODO: Revisar este texto, que foi adaptado do pré-projeto para o tempo passado.]
Este trabalho seguiu uma abordagem de pesquisa quantitativa e experimental. A metodologia foi dividida nas seguintes etapas:

\begin{enumerate}
    \item \textbf{Fonte de Dados:} Foi utilizado o dataset público ``Phishing Dataset for Machine Learning'' \parencite{kaggle_dataset_2022}, cuja base original foi apresentada por \textcite{mandadi2022}. Este conjunto de dados contém um vasto número de amostras e características já extraídas de URLs, as quais são previamente classificadas como phishing ou legítimas.

    \item \textbf{Ferramentas:} O desenvolvimento foi realizado na linguagem Python, com o auxílio das bibliotecas Pandas para manipulação de dados e Scikit-learn \parencite{scikit-learn} para a implementação dos modelos de Machine Learning, além de Matplotlib/Seaborn para a visualização de dados.

    \item \textbf{Tratamento dos Dados:} Foi realizada uma análise exploratória para compreender a distribuição e correlação dos dados. Posteriormente, o conjunto de dados foi dividido em 80\% para o conjunto de treino e 20\% para o conjunto de teste, utilizando a função \textit{train\_test\_split} da biblioteca Scikit-learn.

    \item \textbf{Modelagem e Avaliação:} Foram treinados e avaliados os algoritmos Random Forest e Support Vector Machine. A performance dos modelos foi comparada utilizando as métricas de Acurácia, Matriz de Confusão, Precisão e Recall.
\end{enumerate}


\section{Resultados e Discussão}
\label{sec:resultados}
% [TODO: Este é o capítulo principal que você deve escrever agora. Use o roteiro abaixo.]
Nesta seção, são apresentados e analisados os resultados obtidos a partir da execução da metodologia descrita.

\subsection{Desempenho dos Modelos}
% [TODO: Insira aqui a tabela comparativa com os resultados de Acurácia, Precisão, Recall, etc., para o Random Forest e o SVM.]
% Exemplo de como criar a tabela:
\begin{tabela}[hbt]
    \caption{Resultados comparativos de desempenho dos modelos.}
    \centering
    \begin{tabular}{lcc}
    \specialrule{2pt}{0pt}{1pt}
    \textbf{Métrica} & \textbf{Random Forest} & \textbf{SVM} \\
    \specialrule{1pt}{1pt}{1pt}
    Acurácia & 98.20\% & 86.35\% \\
    Precisão (Phishing) & 0.98 & 0.90 \\
    Recall (Phishing) & 0.98 & 0.82 \\
    Falsos Positivos & 18 & 181 \\
    \specialrule{2pt}{1pt}{0pt}
    \end{tabular}
    \label{tab:resultados}
    \XPT Fonte: O autor (2025)
\end{tabela}

\subsection{Análise Comparativa e Discussão}
Conforme apresentado na Tabela \ref{tab:resultados}, o modelo Random Forest demonstrou uma superioridade significativa em todas as métricas avaliadas.

% [TODO: Escreva a análise aqui. Use um parágrafo para cada ponto abaixo.]
% 1. Compare a acurácia geral.
% 2. Analise a Matriz de Confusão, focando na grande diferença dos Falsos Positivos (18 vs 181) e explique por que esse erro é o mais crítico.
% 3. Compare o Recall para a classe "Phishing" e explique o que significa o SVM ter deixado 18% das ameaças passarem.
% 4. Levante a hipótese do porquê o RF foi melhor (robustez, não necessidade de escalonamento de dados, etc.).


\section{Conclusão}
\label{sec:conclusao}
% [TODO: Escreva a conclusão com base no roteiro abaixo.]

\subsection{Síntese dos Resultados}
% [TODO: Resuma em um parágrafo o que foi feito e o principal achado do trabalho. Ex: "Este trabalho desenvolveu e comparou dois modelos... O modelo Random Forest se mostrou a abordagem mais eficaz, alcançando 98.20% de acurácia..."]

\subsection{Limitações do Trabalho}
% [TODO: Discuta as limitações. A principal é a questão da extração de features simplificada, que foi revelada no teste com a URL do Google. Explique que para uma aplicação real, uma pipeline de extração de features completa seria necessária.]

\subsection{Trabalhos Futuros}
% [TODO: Sugira os próximos passos. Ex: 1. Desenvolvimento da pipeline completa de extração de features. 2. Otimização dos hiperparâmetros do SVM. 3. Teste com outros algoritmos, como redes neurais.]


% --- BIBLIOGRAFIA ---
\printbibliography[title={REFERÊNCIAS}]

\end{document}